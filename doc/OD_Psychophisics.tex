%% This is file `elsarticle-template-1-num.tex',
%%
%% Copyright 2009 Elsevier Ltd
%%
%% This file is part of the 'Elsarticle Bundle'.
%% ---------------------------------------------
%%
%% It may be distributed under the conditions of the LaTeX Project Public
%% License, either version 1.2 of this license or (at your option) any
%% later version.  The latest version of this license is in
%%    http://www.latex-project.org/lppl.txt
%% and version 1.2 or later is part of all distributions of LaTeX
%% version 1999/12/01 or later.
%%
%% Template article for Elsevier's document class `elsarticle'
%% with numbered style bibliographic references
%%
%% $Id: elsarticle-template-1-num.tex 149 2009-10-08 05:01:15Z rishi $
%% $URL: http://lenova.river-valley.com/svn/elsbst/trunk/elsarticle-template-1-num.tex $
%%
\documentclass[preprint,review,12pt]{elsarticle}

%% Use the option review to obtain double line spacing
%% \documentclass[preprint,review,12pt]{elsarticle}

%% Use the options 1p,twocolumn; 3p; 3p,twocolumn; 5p; or 5p,twocolumn
%% for a journal layout:
%% \documentclass[final,1p,times]{elsarticle}
%% \documentclass[final,1p,times,twocolumn]{elsarticle}
%% \documentclass[final,3p,times]{elsarticle}
%% \documentclass[final,3p,times,twocolumn]{elsarticle}
%% \documentclass[final,5p,times]{elsarticle}
%% \documentclass[final,5p,times,twocolumn]{elsarticle}


%%Use this packageto allow for accents 
\usepackage[utf8x]{inputenc}

%% The graphicx package provides the includegraphics command.
\usepackage{graphicx}
%% The amssymb package provides various useful mathematical symbols
\usepackage{amssymb}
%% The amsthm package provides extended theorem environments
%% \usepackage{amsthm}

%% The lineno packages adds line numbers. Start line numbering with
%% \begin{linenumbers}, end it with \end{linenumbers}. Or switch it on
%% for the whole article with \linenumbers after \end{frontmatter}.
\usepackage{lineno}

%% natbib.sty is loaded by default. However, natbib options can be
%% provided with \biboptions{...} command. Following options are
%% valid:

%%   round  -  round parentheses are used (default)
%%   square -  square brackets are used   [option]
%%   curly  -  curly braces are used      {option}
%%   angle  -  angle brackets are used    <option>
%%   semicolon  -  multiple citations separated by semi-colon
%%   colon  - same as semicolon, an earlier confusion
%%   comma  -  separated by comma
%%   numbers-  selects numerical citations
%%   super  -  numerical citations as superscripts
%%   sort   -  sorts multiple citations according to order in ref. list
%%   sort&compress   -  like sort, but also compresses numerical citations
%%   compress - compresses without sorting
%%
%% \biboptions{comma,round}

% \biboptions{}

\journal{Journal of Mathematical Psychology}

\begin{document}

\begin{frontmatter}

%% Title, authors and addresses

\title{Optimal Experimental Design: Psychophysics of change point Detection.}

%% use the tnoteref command within \title for footnotes;
%% use the tnotetext command for the associated footnote;
%% use the fnref command within \author or \address for footnotes;
%% use the fntext command for the associated footnote;
%% use the corref command within \author for corresponding author footnotes;
%% use the cortext command for the associated footnote;
%% use the ead command for the email address,
%% and the form \ead[url] for the home page:
%%
%% \title{Title\tnoteref{label1}}
%% \tnotetext[label1]{}
%% \author{Name\corref{cor1}\fnref{label2}}
%% \ead{email address}
%% \ead[url]{home page}
%% \fntext[label2]{}
%% \cortext[cor1]{}
%% \address{Address\fnref{label3}}
%% \fntext[label3]{}


%% use optional labels to link authors explicitly to addresses:
%% \author[label1,label2]{<author name>}
%% \address[label1]{<address>}
%% \address[label2]{<address>}

\author{Manuel Villarreal and Arturo Bouzas}

\address{Mexico City, Mexico}

\begin{abstract}
%% Text of abstract
\end{abstract}

\begin{keyword}
Optimal Experimental Design \sep Psycophysics \sep Change Point Detection
%% keywords here, in the form: keyword \sep keyword

%% MSC codes here, in the form: \MSC code \sep code
%% or \MSC[2008] code \sep code (2000 is the default)

\end{keyword}

\end{frontmatter}

%%
%% Start line numbering here if you want
%%
\linenumbers

%% main text
\section{Introduction}
\label{S:1}

Experimental design
First we need a research question, then the prolem of designin the experiment arises, how many participants should we test, what are the values of the independent variable that we should use, how many times should we present each of those values, etc. The problem is when 


% Why is is detecting changes important for an organism?
% 
% Change detection in probabilistic series.
% 
% Arising problems with experimental design.

\subsection{Optimal experimental design}

Elements:

Design space: what are the elements of the experimental design that we want to optimize
Utility Function: function that maps points on the design space to the real numbers, this function should reflect the objective of the experiment, for example if we want to discriminate between two cognitive models, the utility function should assign a greater value to an experimental design for which the models give different predictions that to designs for which the predictions of the models are indistinguishable from one another.

% EXAMPLE CITING Maecenas \cite{Smith:2012qr} fermentum \cite{Smith:2013jd} %

%EXAMPLE ITMES and  ENUMERATING
%\begin{itemize}
%\item Bullet point one
%\item Bullet point two
%\end{itemize}
%\begin{enumerate}
%\item Numbered list item one
%\item Numbered list item two
%\end{enumerate}

%%%% EXAMPLE TABLE
%\begin{table}[h]
%\centering
%\begin{tabular}{l l l}
%\hline
%\textbf{Treatments} & \textbf{Response 1} & \textbf{Response 2}\\
%\hline
%Treatment 1 & 0.0003262 & 0.562 \\
%Treatment 2 & 0.0015681 & 0.910 \\
%Treatment 3 & 0.0009271 & 0.296 \\
%\hline
%\end{tabular}
%\caption{Table caption}
%\end{table}

%%%% EXAMPLE Labeled Equiation
%\begin{equation}
%\label{eq:emc}
%e = mc^2
%\end{equation}

\section{Optimal Experimental Design: Example}
\label{S:2}

Why is is detecting changes important for an organism?

Change detection in probabilistic series.

Arising problems with experimental design.

Research question and its statistichal interpretation

Assumtion about the relationship between a subjects response the dependent variable under study

Design space for this problem and how to reduce the dimensionality of the space by assuming experimental constraints.

Utility function and its relationship with the objective of the experiment

Arising problems with utility function and the proposed response function. Bayesian solution, assigning a prior distribution to the parameters, the less research in a field the more difficult it is to assign an informative prior, however, we could use other cognitive models in order to propose a prior distribution. 
 % EXAMPLE Reference of a labeled Section \ref{S:1}
\subsection{Using a model to generate prior distributions}

Using the prior distribution, the utility function and the definition of a design space we can otimize the experimental design in this case we are looking for \begin{math}\mathbf{\delta\theta}^{*}\end{math} that maximizes the following equation:

\begin{equation}
U(\mathbf{\delta\theta}^{*})=\max_{\delta\theta} \int_{\mathbf{\beta}} log(det(I(\beta|\mathbf{\delta\theta}))) \pi(\mathbf{\beta}) d\mathbf{\beta}
\end{equation}

The previous integral can be approximated via Monte Carlo sampling

\section{Results}
\label{S:3}

\subsection{Consruction of the prior distribution}

Prior over model parameters(Gallistel et al 2014)
Results
Constructing the prior: we take a multivariate normal distribution with mean and covariance equal to the unbiased estimators for both parameters.


\subsection{Optimal design}

Aproximating the utility function (integral) throught Monte Carlo simulation 
Utility aproximation for 2 Design points 

the approximation returns a smooth curve over the 2 point design space.

\section{Discussion}
\label{S:3}

Optimal design for the example
Properties of the most useful points (they land on the points of the curve where the steepness changes most dramatically)

Advantages of Optimal Design

Using models to generte prior distributions.

%% The Appendices part is started with the command \appendix;
%% appendix sections are then done as normal sections
%% \appendix

%% \section{}
%% \label{}

%% References
%%
%% Following citation commands can be used in the body text:
%% Usage of \cite is as follows:
%%   \cite{key}          ==>>  [#]
%%   \cite[chap. 2]{key} ==>>  [#, chap. 2]
%%   \citet{key}         ==>>  Author [#]

%% References with bibTeX database:

\bibliographystyle{model1-num-names}
\bibliography{sample.bib}

%% Authors are advised to submit their bibtex database files. They are
%% requested to list a bibtex style file in the manuscript if they do
%% not want to use model1-num-names.bst.

%% References without bibTeX database:

% \begin{thebibliography}{00}

%% \bibitem must have the following form:
%%   \bibitem{key}...
%%

% \bibitem{}

% \end{thebibliography}


\end{document}

%%
%% End of file `elsarticle-template-1-num.tex'.